\documentclass[11pt]{elegantbook}

\title{Introduction to Mathematical Analysis}
\subtitle{Math 521}

\author{Chris Cai/ Linrong Cai}
\institute{University of Wisconisn Madison}
\date{Nov 18, 2022}
\version{1.0}
\bioinfo{Instructor}{Jordan Ellenberg}

\extrainfo{Mathematical Analysis is as extensive as nature herself.}

\cover{cover.png}

% modify the color in the middle of titlepage
\definecolor{customcolor}{RGB}{32,178,170}
\colorlet{coverlinecolor}{customcolor}
\usepackage{cprotect}

\addbibresource[location=local]{reference.bib} % bib

\begin{document}

\maketitle

\frontmatter
\tableofcontents

\mainmatter
\chapter{Basics}

\section{Set}
\begin{definition}
    A set is a collection of objects called elements.
\end{definition}
\begin{definition}
    The Cartesian product $ A\times B $ is the set whose elements are odered pairs $(a, b)$ with $ a \in A $, $ b \in B $.
    Particularly, $ A \times A $: $\{(ai, aj) \ \vert \  ai, aj \in A \}$. Often $A^2$
\end{definition}

\begin{quotation}
    Henri Poincaré: Mathematics is the art of giving the same name to different things.
\end{quotation}

\begin{definition}
    A function $f$ from A to B is a subset $f \subset A \times B $, with the properties if $(a, b)\in f $ and $(a, b') \in f$ then $b = b'$.  We write $f(a)= b $ for $(a, b ) \in f $ means A function $f$ from A to B. 
    $$ f: A \rightarrow  B \qquad \qquad A \xrightarrow{f} B $$
\end{definition}

\begin{definition}
    We say a function $f: A \rightarrow B $ is injective if $f(a) = f(a') \Rightarrow a = a'$.
    We say it is surjective if $\forall  b, \exists a \mid f(a) = b$.
\end{definition}

\begin{corollary}
    Injective: if for every $b \in B $, there is at most one $a \in A$ with $f(a) = b $ \\
    Surjective: if for every $b \in B $, there is at least one $a \in A$ with $f(a) = b $
\end{corollary}

\begin{note}
    If f is both injective and surjective for every $b \in B $. There is exactly one $a \in A$ such that $f(a)= b$, and we called this function is bijective
\end{note}

\begin{definition}
    If $G \subset B$, then the inverse image $f^{-1}(G)$ of $G$ under $f$ is defined as:
    $$\{x \in A : f(x)\in G\}$$
\end{definition}

If $f$ is bijective then there is a new function $f^{-1}: B \rightarrow A$ 

\chapter{Functions}

\section{Connectness}

\begin{definition}{Topologist Definition}
    $X$ is connected if there are no 2 non-empty open stes $U_1, U_2\subseteq X$ with $ U_1 \cap U_2 = \emptyset$ and  $ U_1 \cup U_2$ = X
\end{definition}

\begin{note}
When $U_2 = U_1^c$, and they satisfy the above condition, $U_2$ is closed and open 
\end{note}

\begin{definition}{Analyst Definition}
    $X$ is connected if there is no continuous surjective function $f: X \rightarrow \{0, 1\}$ 
\end{definition}

\begin{proof}{:}
    If $f: X \rightarrow \{0, 1\}$ surjective and continuous. Let $U_1 = f^{-1}(0)$, $U_2 = f^{-1}(1)$ are non-empty and open since $f$ is surjective and continuous.

    As a result, $U_1$ and $U_2$ are disjoint and cover $X$.
\end{proof}

\begin{equation}
    $$Hello$$
\end{equation}

\begin{corollary}
    Cantor set is disconnected, $\mathbb{Q}$ is disconnected.
\end{corollary}
\begin{proof}
    Map first $\frac{1}{3}$ of Cantor set to $\{0\}$, last $\frac{1}{3}$ to $\{1\}$. $\mathbb{Q}$ is separeated by $\sqrt{2}$.
\end{proof}
\section{Sequence of function}


\end{document}